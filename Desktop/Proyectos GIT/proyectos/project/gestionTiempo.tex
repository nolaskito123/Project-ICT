\chapter{Gestión del tiempo}
%\paragraph{}WRITE HERE

\section{Descripción de actividades}


\subsection{Obtención de planos arquitectónicos}
\begin{itemize}
\item \textbf{Identificador: }1.1.1 A.
\item \textbf{Paquete de trabajo: }Planos arquitectónicos.
\item \textbf{Descripción: }Mediante una citación telemática o telefónica, se concierta una reunión con el Arquitecto de la edificación. En dicha reunión se obtendrán los planos arquitectónicos del edificio en fase de finalización de obra, dado que en esta fase no se modificará la edificación y se podrá ejecutar la instalación de la ICT.
\item \textbf{Responsable: }Director.
\item \textbf{Requisitos de recursos: }
\item \textbf{Estimación de duración: }1 día.
\item \textbf{Secuencia de actividades: }Esta actividad no necesita de otra para ser iniciada, sino que es la primera actividad de la que procederán los siguientes paquetes de trabajo.
\end{itemize}

\subsection{Álex1}
\begin{itemize}
\item \textbf{Identificador: }
\item \textbf{Paquete de trabajo: }
\item \textbf{Descripción: }
\item \textbf{Responsable: }
\item \textbf{Requisitos de recursos: }
\item \textbf{Estimación de duración: }
\item \textbf{Secuencia de actividades: }
\end{itemize}

\subsection{Álex2}
\begin{itemize}
\item \textbf{Identificador: }
\item \textbf{Paquete de trabajo: }
\item \textbf{Descripción: }
\item \textbf{Responsable: }
\item \textbf{Requisitos de recursos: }
\item \textbf{Estimación de duración: }
\item \textbf{Secuencia de actividades: }
\end{itemize}

\subsection{Obtener medidas de potencia de señal}
\begin{itemize}
\item \textbf{Identificador: }1.1.2 D.
\item \textbf{Paquete de trabajo: }Evaluación de emplazamiento.
\item \textbf{Descripción: }Vamos a la ubicación, donde se va a realizar el proyecto, evaluamos todos aspectos necesarios para realizar los cálculos previos, uno de estos aspectos es obtener las medidas de potencia de señal. Medimos la potencia de señal recibida en varios puntos del emplazamiento y estudiaremos cuál es el mejor sitio para colocar las antenas de la instalación RTV.
\item \textbf{Responsable: }Ingenieros y los instaladores.
\item \textbf{Requisitos de recursos: }Para esta actividad será necesario un medidor de radiación electromagnética para observar la calidad de señal recibida y sus interferencias. Los ingenieros escogerán 5 lugares, previamente estudiados, para medir. Los instaladores realizarán las medidas en dichos puntos.
\item \textbf{Estimación de duración: }3-4 días.
\item \textbf{Secuencia de actividades: }Para hacer esta actividad es necesario que se haya obtenido los planos arquitectónicos(paquete de trabajo 1.1.1). Tras esta actividad, ya se puede hacer los cálculos previos (paquete de trabajo 1.2.1).
\end{itemize}

\subsection{Entregar el proyecto a una entidad verificadora}
\begin{itemize}
\item \textbf{Identificador: }1.3.1 A.
\item \textbf{Paquete de trabajo: }COITT.
\item \textbf{Descripción: }El director de obra entregará el proyecto a una entidad verificadora, como el  Colegio Oficial de Ingenieros Técnicos en Telecomunicaciones(COITT). Esta entidad irá informando si hay que realizar cambios en el proyecto y,finalmente, dará respuesta si el proyecto es verificado o no.
\item \textbf{Responsable: }Director de obra.
\item \textbf{Requisitos de recursos: }Es necesario tener redactado el proyecto.
\item \textbf{Estimación de duración: }
\item \textbf{Secuencia de actividades: }Esta actividad se realiza cuando se haya acabado de redactar el proyecto(paquete de trabajo 1.2). Tras esta actividad se puede proceder a la ejecución e instalación(paquete de trabajo 1.4).
\end{itemize}

\subsection{Instalación del cableado}
\begin{itemize}
\item \textbf{Identificador: }1.4.3 A.
\item \textbf{Paquete de trabajo: }Instalación del cableado.
\item \textbf{Descripción: }...
\item \textbf{Responsable: }Instaladores.
\item \textbf{Requisitos de recursos: }... 
\item \textbf{Estimación de duración: }...
\item \textbf{Secuencia de actividades: }Esta actividad viene precedida de la tarea 1.4.2. Tras esta actividad, se realizarán los paquetes 1.4.4 y 1.5.
\end{itemize}

\subsection{Resolver dudas en el proceso de intalación}
\begin{itemize}
\item \textbf{Identificador: }1.4.5.1 B.
\item \textbf{Paquete de trabajo: }Supervisión.
\item \textbf{Descripción: }Los ingenieros estarán presentes en el proceso de instalación para resolver las dudas que puedan surgirles a los instaladores. El director visitará la obra periódicamente (dos veces a la semana) por si su ayuda fuese necesaria. En caso de requerir la opinión del cliente para ciertos aspectos del montaje de la instalación, será el director el encargado de comunicarse con este para conocer sus criterios y preferencias.
\item \textbf{Responsable: }Director e Ingenieros.
\item \textbf{Requisitos de recursos: }Ninguno (la presencia de los responsables en el lugar de la instalación es suficiente).
\item \textbf{Estimación de duración: }7 semanas.
\item \textbf{Secuencia de actividades: } Esta actividad debe realizarse simultáneamente al proceso de instalación, lo que representa el conjunto de todas las actividades de (1.4).
\end{itemize}

\subsection{Realización del manual de usuario}
\begin{itemize}
\item \textbf{Identificador: }1.5 C.
\item \textbf{Paquete de trabajo: }Documentos finales.
\item \textbf{Descripción: }El director, con ayuda del jefe de proyecto si fuese necesario, redactará el manual de usuario de la ICT que se ha realizado, siguiendo las indicaciones del Anexo VI de la Orden ITC/1644/2011. Dicho manual deberá incluir:
\begin{enumerate}
\item Identificación del edificio
\item Objetivo del documento
\item Introducción
\item Esquema de la instalación efectuada
\item Resumen de servicios instalados
\item Descripción de la instalación interior de usuario
\item Servidumbres
\item Garantía de la ICT
\item Documentación de las Instalaciones de Telecomunicación de la Edificación (ICT)
\item Recomendaciones de mantenimiento para las instalaciones
\end{enumerate}
Posteriormente a su redacción, se entregarán dos copias del manual de usuario (una en catalán y una en castellano) a la propiedad del edificio.
\item \textbf{Responsable: }Director
\item \textbf{Requisitos de recursos: }Se requerirá del uso de un PC con editor de texto y de una impresora con papel y tinta suficientes
\item \textbf{Estimación de duración: }2 días.
\item \textbf{Secuencia de actividades: }Esta actividad deberá realizarse después de haber terminado todas las actividades de montaje (1.4.2) y canalización (1.4.3) de la instalación. También es recomendable que se haya terminado la actividad de realización de pruebas (1.4.4 A).
\end{itemize}
